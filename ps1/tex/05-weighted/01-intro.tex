\item \subquestionpoints{10} Consider a linear regression problem in which
we want to ``weight'' different training examples differently.  Specifically,
suppose we want to minimize
%
\begin{equation*}
	J(\theta) = \frac{1}{2} \sum_{i=1}^m w^{(i)}
		\left(\theta^Tx^{(i)} - y^{(i)}\right)^2.
\end{equation*}
%
In class, we worked out what happens for the case where all the weights (the
$w^{(i)}$'s) are the same. In this problem, we will generalize some of those
ideas to the weighted setting.
\begin{enumerate}
	\item \subquestionpoints{2} Show that $J(\theta)$ can also be written
    %
    \begin{equation*}
    J(\theta) = (X\theta - {y})^T W (X\theta - {y})
    \end{equation*}
    %
    for an appropriate matrix $W$, and where $X$ and ${y}$ are as
    defined in class. Clearly specify the value of each element of the matrix
    $W$.

	\item \label{item:lwr-solution} \subquestionpoints{4} If all the $w^{(i)}$'s
    equal 1, then we saw in class that the normal equation is
    %
    \begin{equation*}
    X^TX\theta = X^T{y},
    \end{equation*}
    %
	and that the value of $\theta$ that minimizes $J(\theta)$ is given by
	$(X^TX)^{-1}X^T{y}.$
	By finding the derivative $\nabla_\theta J(\theta)$ and setting that to zero,
	generalize
	the normal equation to this weighted setting, and give the new value of
	$\theta$ that minimizes
    $J(\theta)$ in closed form as a function of $X$, $W$ and ${y}$.

	\item \subquestionpoints{4} Suppose we have a dataset
	$\{(x^{(i)}, y^{(i)});\, i=1\ldots,m\}$ of $m$ independent examples, but
    we model the $y^{(i)}$'s as drawn from conditional distributions with
    different levels of variance $(\sigma^{(i)})^2$. Specifically, assume the
    model
    %
    \begin{equation*}
		p(y^{(i)} | x^{(i)} ; \theta) = \frac{1}{\sqrt{2\pi}\sigma^{(i)}} \exp\left(-
		\frac{(y^{(i)} - \theta^Tx^{(i)})^2}{2(\sigma^{(i)})^2}\right)
	\end{equation*}
    %
    That is, each $y^{(i)}$ is drawn from a Gaussian distribution with mean
    $\theta^Tx^{(i)}$ and variance $(\sigma^{(i)})^2$ (where the
    $\sigma^{(i)}$'s are fixed, known, constants). Show that finding the
    maximum likelihood estimate of $\theta$ reduces to solving a weighted
	linear regression problem.  State clearly what the $w^{(i)}$'s are in terms of
    the $\sigma^{(i)}$'s.
\end{enumerate}

\ifnum\solutions=1
  \begin{answer}
    \begin{enumerate}
    	\item
    	    	Obviously,
\begin{align*}
J(\theta) =& \frac{1}{2} 	\begin{bmatrix}
\theta^Tx^{(1)}-y^{(1)}, & \cdots, & \theta^Tx^{(m)}-y^{(m)}
\end{bmatrix} \begin{bmatrix}
w^{(1)}& & \\
& \ddots &\\
& & w^{(m)}
\end{bmatrix}\begin{bmatrix}
\theta^Tx^{(1)}-y^{(1)}\\
\vdots\\
\theta^Tx^{(m)}-y^{(m)}
\end{bmatrix}\\
= &(\theta^TX^T-y^T)W(X\theta -y) \qquad \quad(\mbox{with}\  W = \frac{1}{2}
\mbox{diag}(w^{(1)},...,w^{(m)})\ )\\
= &(X\theta-y)^TW(X\theta-y)
\end{align*}
\item
\[ \frac{\partial J}{\partial \theta}=\frac{\partial }{\partial \theta}(\theta^TX^TWX\theta-2\theta^TX^TWy+y^TWy)=2X^TWX\theta-2X^TWy \]
so if we set $ w^{(i)}=2,\frac{\partial J}{\partial \theta}=0 $ then $ W=\frac{1}{2}I_{(m)}$ and $ \theta=(X^TX)^{-1}X^T{y}. $
\item
\[ p(y|X;\theta) = \prod_{i=1}^{m}p(y^{(i)}|x(^{i});\theta) = (2\pi)^{(-m/2)}(\prod_{i=1}^{m}\sigma^{(i)})^{-1}\exp\left( -\sum_{i=1}^{m}\frac{(y^{(i)}-\theta^Tx^{(i)})^2}{2(\sigma^{(i)})^2}\right)  \]
let $ w^{(i)} = \frac{1}{(\sigma^{(i)})^2}  $ and use the result in ii. to solve the regression problem.
    \end{enumerate}

\end{answer}

\fi
